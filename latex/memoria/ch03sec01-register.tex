\section{Caso de Uso: Registrarse}

\subsection{Análisis y diseño}

\subsection{Implementación}

\subsection{Informe de pruebas}

{\small
\begin{tabular}{r|l}
Nombre del \textit{tester} & Antonio Gómez Poblete \\
Fecha de asignación & 27 de enero de 2011 \\
Fecha de finalización & 30 de enero de 2011 \\
Código bajo prueba & \texttt{UserManager::registerUser}
\end{tabular}
}

\subsubsection{Pruebas de desarrollo}

A continuación se detallarán las pruebas de desarrollo (Pruebas unitarias con \textit{Junit}).

Lista de los valores de prueba para cada atributo.
El criterio elegido para todos los valores de prueba (test data) ha sido: Añadir valores interesantes propensos a error (Conjetura de error).

\begin{itemize}
\item \textbf{\texttt{login}}
\subitem \textit{null}
\subitem Cadena vacía
\subitem Usuario existente: \texttt{"JorgeCA"}
\subitem Usuario no existente: \texttt{"LuisAn"}

\item \textbf{\texttt{passwd}}
\subitem \textit{null}
\subitem Cadena vacía
\subitem Cadenas no vacías: \texttt{"jorge"}, \texttt{"luis"}

\item \textbf{\texttt{email}}
\subitem \textit{null}
\subitem Cadena vacía
\subitem Valores incorrectos: \texttt{"jorge"}, \texttt{"jorge@"}, \texttt{"jorge@gmail"}, \texttt{"jorge@gmail."}
\subitem Valores correctos: \texttt{"jorge.colao@gmail.com"}, \texttt{"luis@gmail.com"}
\end{itemize}

La estrategia para obtener los casos de prueba elegida ha sido
\textit{each choice} (Test 1 a 3 y 8 a 12 ) y añadir algunos casos
interesantes (Test 4 a 7).

Para elegir tanto los valores de prueba como los casos de pruebas se ha analizado el comportamiento de este caso de uso (caja blanca). Tras este análisis se ha llegado a la conclusión de que hay en dos lugares en los que se pueden encontrar errores: Si al método registerUser se le da un parámetro erróneo (throw InvalidArgumentException), o si el nombre del usuario ya existe en el servidor (throw UserAlreadyExistsException). La primera comprobación es local, por ello si se lanza la excepcion InvalidArgumentException no se hará nada en el servidor y no se analizará si el usuario ya existe o no.

Además, si observamos las comprobaciones que se hacen en el cliente se ve que si login == null, independientemente del valor de los demás atributos se lanzará  InvalidArgumentException, por esa razón es independiente el valor de los demas atributos, lo que hace que muchos casos de prueba no sean necesarios de probar. Por otro lado, para probar qué pasa si el email es \textit{null} el resto de los parámetros deberán ser correctos, lo que añade algunos casos interesantes. Además de con \textit{null} lo comentado en este parrafo sucede con el resto de los parámetros.


Teniendo en cuenta las consideraciones anteriormente mencionadas, la tabla completa de los casos de prueba y los resultados esperados son:

\newpage
{\footnotesize
\begin{longtable}[c]{l|c|c|c|}
 & \textbf{Valores de Prueba} & \textbf{Objetivo del test} & \textbf{Resultado esperado} \\
\hline \hline
\endhead

Test1 & (vacío, vacío, vacío)  & login & InvalidArgument\\
Test2 & (null, null, null) & login & InvalidArgument\\
Test3 & (JorgeCA, jorge, jorge.colao@gmail.com) & Usuario ya existente & UserAlreadyExistst\\
Test4 & (JorgeCA, vacío, jorge.colao@gmail.com) & passwd   & InvalidArgument\\
Test5 & (JorgeCA, null, jorge.colao@gmail.com) & passwd   & InvalidArgument\\
Test6 & (JorgeCA, jorge, vacío) & email   & InvalidArgument\\
Test7 & (JorgeCA, jorge, null) & email   & InvalidArgument\\
Test8 & (JorgeCA, jorge, jorge) & email & InvalidArgument\\
Test9 & (JorgeCA, jorge, jorge@) & email & InvalidArgument\\
Test10 & (JorgeCA, jorge, jorge@gmail) & email & InvalidArgument\\
Test11& (JorgeCA, jorge, jorge@gmail.) & email  & InvalidArgument\\
Test12& (LuisAn, luis, luis@gmail.com) & email  &  El usuario \\
& &  & queda registrado\\
\hline
\end{longtable}
}

\subsubsection{Pruebas exploratorias}

A continuación se muestra un guión de pruebas exploratorias para cada uno de los escenarios posibles:
\begin{enumerate}
\item Todos los datos son válidos y el usuario no existe.
	\begin{itemize}
	\item Acciones previas: El servidor debe estar funcionando y no tener un usuario con el nombre LuisAn.
	\item Acciones de prueba : En la ventana de inicio pinchamos en registrarnos y rellenamos el diálogo que nos aparecerá con los siguientes valores: LuisAn, luis, luis@gmail.com.
	\item Resultados esperados : Tras pulsar en aceptar  desaparecerá el diálogo de registro y en la parte superior de la ventana de inicio nos aparecerá en verde \emph {Usuario : LuisAn registrado"}.
	\end{itemize}
\item Todos los datos son válidos, pero el usuario existe.
	\begin{itemize}
	\item Acciones previas: El servidor debe estar funcionando y no tener un usuario con el nombre JorgeCA.
	\item Acciones de prueba : En la ventana de inicio pinchamos en registrarnos y rellenamos el diálogo que nos aparecerá con los siguientes valores: JorgeCA, jorge, jorge.colao@gmail.com.
	\item Resultados esperados : Tras pulsar en aceptar desaparecerá el diálogo de registro y en la parte superior de la ventana de inicio nos aparecerá en rojo \emph {El servidor indica: Error en el registro}.
	\end{itemize}

\item Algún dato introducido en el diálogo no es válido, independientemente de que el usuario exista o no.
	\begin{itemize}
	\item Acciones previas: El servidor debe estar funcionando.
	\item Acciones de prueba : En la ventana de inicio pinchamos en registranos y rellenamos el diálogo que nos aparecerá con los siguientes valores: JorgeCA, jorge, jorge@gmail..
	\item Resultados esperados : Tras pulsar en aceptar desaparecerá el diálogo de registro y aparecerá un panel de error mostrando \emph{Algún argumento es erroneo}. Despues de pulsar aceptar en el panel, volverá a aparecer el diálogo. Esto sucederá hasta que los datos sean correctos o demos a cancelar.
	\end{itemize}
\end{enumerate}
