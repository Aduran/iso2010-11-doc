\section{Registrarse}

A continuación se muestra un guión de pruebas exploratorias para cada uno de los escenarios posibles:
\begin{enumerate}
\item Todos los datos son válidos y el usuario no existe.
	\begin{itemize}
	\item Acciones previas: El servidor debe estar funcionando y no tener un usuario con el nombre LuisAn.
	\item Acciones de prueba : En la ventana de inicio pinchamos en registrarnos y rellenamos el diálogo que nos aparecerá con los siguientes valores: LuisAn, luis, luis@gmail.com.
	\item Resultados esperados : Tras pulsar en aceptar  desaparecerá el diálogo de registro y en la parte superior de la ventana de inicio nos aparecerá en verde \emph {Usuario : LuisAn registrado"}.
	\end{itemize}
\item Todos los datos son válidos, pero el usuario existe.
	\begin{itemize}
	\item Acciones previas: El servidor debe estar funcionando y no tener un usuario con el nombre JorgeCA.
	\item Acciones de prueba : En la ventana de inicio pinchamos en registrarnos y rellenamos el diálogo que nos aparecerá con los siguientes valores: JorgeCA, jorge, jorge.colao@gmail.com.
	\item Resultados esperados : Tras pulsar en aceptar desaparecerá el diálogo de registro y en la parte superior de la ventana de inicio nos aparecerá en rojo \emph {El servidor indica: Error en el registro}.
	\end{itemize}

\item Algún dato introducido en el diálogo no es válido, independientemente de que el usuario exista o no.
	\begin{itemize}
	\item Acciones previas: El servidor debe estar funcionando.
	\item Acciones de prueba : En la ventana de inicio pinchamos en registranos y rellenamos el diálogo que nos aparecerá con los siguientes valores: JorgeCA, jorge, jorge@gmail..
	\item Resultados esperados : Tras pulsar en aceptar desaparecerá el diálogo de registro y aparecerá un panel de error mostrando \emph{Algún argumento es erroneo}. Despues de pulsar aceptar en el panel, volverá a aparecer el diálogo. Esto sucederá hasta que los datos sean correctos o demos a cancelar.
	\end{itemize}
\end{enumerate}
