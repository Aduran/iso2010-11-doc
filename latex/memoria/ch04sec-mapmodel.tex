\section{Clase MapModel}

{\small
\begin{tabular}{r|l}
Nombre del \textit{tester} & Antonio Gómez Poblete \\
Fecha de asignación & 21 de febrero de 2011 \\
Fecha de finalización & 25 de febrero de 2011 \\
Código bajo prueba & \texttt{MapModel}
\end{tabular}
}

A diferencia de otras clases como los gestores, la clase \texttt{MapModel} únicamente contiene métodos para implementar el patrón MVC, no  implementa casi ninguna otra funcionalidad . Estos métodos son en su mayoría para acceder a los datos que el modelo proporciona y no filtran los atributos de entrada (en ocasiones no tienen) como pueden hacerlo otras clases de dominio. Por esta razón para comprobar esta clase se va a hacer usa de técnicas de caja blanca , intentando cubrir una máxima cobertura del código.  

A continuación se van a ir mencionando los distintos test que se han realizado a esta clase para obtener un cobertura de un 99 \%, teniendo en cuenta que como precondición debe darse que exista un mapModel (recién creado) y dos jugadores (\texttt{Antonio} y \texttt{Ambrosio}) sin ningún territorio.  


\subsection{MapModelTest::testMapModel} 

En este test se comprueba que el constructor de la clase \texttt{MapModel} funciona adecuadamente. Para ello se pregunta por cada uno de los territorios si  son distintos a \texttt{null} y si tienen el identificador que les corresponde.

\subsection {MapModelTest::testSetData}

En este test se comprueba que si los territorios son asignados nuevamente, la información es correcta.

Este test además de probar la función \texttt{setData}, prueba también el método \texttt{updateTerritory} ya que este es llamado desde \texttt{setData}.

\subsection {MapModelTest::testUpdateTerritory}

En este test se actualiza un territorio existente, para ello se le asigna un jugador (\texttt{Antonio}) y se comprueba que ningún país tienen un jugador salvo el país de \texttt{Antonio}. 

\subsection {MapModelTest::testgetColumnName}

En este test se comprueba que \texttt{testgetColumnName} devuelve el nombre correcto para todos los casos.

\subsection {MapModel::getValueAt}

A diferencia de  los métodos anteriormente mencionados, para probar \texttt{getValueAt} se ha aplicado también un enfoque de caja negra.

El criterio elegido para todos los valores de prueba (test data) ha sido: Añadir valores límite y valores interesantes propensos a error (Conjetura de error)  
\begin{itemize}
\item \textbf{\texttt{rowIndex}}
\subitem  Valores fuera de rango: \texttt{map.getRowCount()}, \texttt{-1}.
\subitem \texttt{0}

\item \textbf{\texttt{columnIndex}}
\subitem  Valores fuera de rango: \texttt{map.getColumnCount()}, \texttt{-1}.
\subitem \texttt{0}
\end{itemize}

La estrategia para obtener los casos de prueba elegida ha sido
\textit{each choice}  y añadir algunos casos interesantes .


Teniendo en cuenta las consideraciones anteriormente mencionadas, la tabla completa de los casos de prueba y los resultados esperados son:

{\footnotesize
\begin{longtable}[c]{lcc}
 & \textbf{Valores de Prueba}  & \textbf{Resultado esperado} \\
\hline \hline
\endhead
Test1 & (-1, 0)  & null\\
Test2 & (\texttt{map.getRowCount()}, 0) &  null\\
Test3 & (0, -1)  & null\\
Test4 & (0, \texttt{map.getColumnCount()}) &  null\\
\hline
\end{longtable}
}

\subsubsection {MapModelTest::testgetValueAt5}

Este test ha sido creado para completar los casos de pruebas antes mencionados y así obtener una mayor cobertura. Además los anteriores test esperaban el mismo resultado (todos producían error). 

En este test los dos jugadores (\texttt{Antonio} y \texttt{Ambrosio}) son asignados a dos territorios. Luego se pregunta por la información de dos territorios (14, 6) y se comprueba que al no tener espías solo se mostrará \texttt{¿?} y el nombre del país.

Para finalizar se añade un espía al jugador \texttt{Antonio} que es el \texttt{selfPlayer} y se comprueba que ahora si es posible saber la información del país (14) también se comprueba que  todos los atributos de este son correctos.





