\section{Caso de Uso: Login}

\subsection{Análisis y diseño}

\subsection{Implementación}

\subsection{Informe de pruebas}

{\small
\begin{tabular}{r|l}
Nombre del \textit{tester} & Angel Durán Izquierdo \\
Fecha de asignación & 27 de enero de 2011 \\
Fecha de finalización & 2 de febrero de 2011 \\
Código bajo prueba & \texttt{UserManager::createSession}
\end{tabular}
}

\subsubsection{Pruebas de desarrollo}

A continuación se detallarán las pruebas de desarrollo (Pruebas unitarias con \textit{JUnit}).

Lista de los valores de prueba para cada atributo.
El criterio elegido para todos los valores de prueba (test data) ha sido: Añadir valores interesantes propensos a error (Conjetura de error).

\begin{itemize}
\item \textbf{\texttt{login}}
\subitem \textit{null}
\subitem Cadena vacía
\subitem Usuario existente: \texttt{Angel}
\subitem Usuario no existente: \texttt{ADuran}
\subitem Nombre de usuario registrado pero introducido con mayusculas: \texttt{ADuran}
\subitem Usuario existente con carácteres especiales: \texttt {Angel\&Duran}
\subitem Usuario existente con numeros: \texttt{1111, -1}

\item \textbf{\texttt{passwd}}
\subitem \textit{null}
\subitem Cadena vacía
\subitem Cadenas no vacías: \texttt{angel}
\subitem Cadenas con carácteres especiales: \texttt{22\&22}
\subitem Cadenas con numeros: \texttt{2222}

\end{itemize}

La estrategia para obtener los casos de prueba elegida ha sido
\textit{each choice} eligiendo los casos mas interesantes.

Para elegir tanto los valores de prueba como los casos de pruebas se ha analizado el comportamiento de este caso de uso (caja blanca). Tras este análisis se ha llegado a la conclusión de que hay en dos lugares en los que se pueden encontrar errores: Si al método createSession se le da un parámetro erróneo (throw InvalidArgumentException), o si el nombre del usuario no se ha registrado en el servidor (throw UserAlreadyExistsException).

Por otro lado se comprueba si al crear una sesion cuando ya se existía una, la anterior se cierra correctamente.


Teniendo en cuenta las consideraciones anteriormente mencionadas, la tabla completa de los casos de prueba y los resultados esperados son:

{\footnotesize
\begin{longtable}[c]{lccc}
 & \textbf{Valores de Prueba} & \textbf{Objetivo del test} & \textbf{Resultado esperado} \\
\hline \hline
\endhead

Test1 & (vacío, vacío)  & login-pass & InvalidArgument\\
Test2 & (Aduran, vacío) & pass & InvalidArgument\\
Test3 & (vacío, angel) & login & UserAlreadyExistst\\
Test4 & (Aduran, angel) & login-pass   & Sesión creada correctamente\\
Test5 & (null, angel) & login   & InvalidArgument\\
Test6 & (Aduran, null) & pass   & InvalidArgument\\
Test7 & (null, null) & login-pass   & InvalidArgument\\
Test8 & (ADuran, angel) & login & WrongLoginException\\
Test9 & (Aduran, Angel) & pass & WrongLoginException\\
Test10 & (1111, 22\&22) & login-pass & Sesión creada correctamente\\
Test11& (-1, 2222) & login-pass  & Sesión creada correctamente\\
Test12& (Angel\&Duran, angel) & login  &  Sesión creada correctamente \\
Test13 & (Aduran y ricki, angel y ricki) & Comportamiento createSession & Sesión creada correctamente\\
Test14 & (Aduran, angel) & Comportamiento createSession & RemoteException\\
Test15 & (Aduran, angel) & Comportamiento createSession & Sesión cerrada correctamente\\
Test16 & (Aduran, angel) & Comportamiento createSession & Sesión creada correctamente\\

\hline
\end{longtable}
}
