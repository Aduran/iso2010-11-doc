\section{Resumen del proyecto}

\subsection{Propósito, alcance y objetivos}

\textit{La Conquista del Mundo} es un juego de estrategia en tiempo real que es
jugado sobre un tablero con regiones que representan el mundo. Este juego sigue
una arquitectura cliente-servidor y hace uso de un motor de comunicaciones de
red para conseguirlo.

El propósito de este proyecto es pues realizar todo el ciclo de desarrollo hasta
conseguir implementar un cliente para ordenador de \textit{La Conquista del
Mundo}.

Se consideran dentro del alcance del proyecto no sólo las actividades de
desarrollo del producto, sino también las relacionadas con conseguir un entorno
de desarrollo y ejecución adecuados.

El objetivo principal del proyecto será la realización de la práctica de manera
satisfactoria antes de la fecha de entrega. Además habrá otros objetivos
secundarios, en su mayor parte formativos, como el conocimiento y puesta en
práctica del Proceso Unificado de Desarrollo o el uso de repositorios software
en proyectos con varios integrantes.

\subsection{Asunciones y restricciones}

Para el desarrollo de este proyecto no será necesario ningún tipo de hardware
especial; los ordenadores personales de los propios desarrolladores serán
suficiente.

Las herramientas software necesarias estarán a disposición de los
desarrolladores a coste cero, bien porque sean herramientas libres o porque la
universidad proporcione las licencias correspondientes.

Se entiende que los desarrolladores son estudiantes, y que por tanto el tiempo
que podrán dedicar al proyecto es limitado. Este hecho debe estar reflejado en
la carga semanal de trabajo.

Por el caracter formativo del proyecto, cada desarrollador deberá adoptar al
menos una vez durante el desarrollo del proyecto cada uno de los siguientes
roles: analista, diseñador, programador y probador.

Este proyecto depende del proyecto encargado del servidor de la aplicación. Las
comunicaciones con el servidor se realizarán mediante la tecnología RMI y la
interfaz de comunicaciones será definida por el equipo del servidor.

Se deberá entregar a los destinatarios de este proyecto una máquina virtual
para VirtualBox que contendrá todo el desarrollo del proyecto así como un
entorno donde ejecutar la aplicación. El sistema operativo de esa máquina
virtual queda a elección del grupo.

El número de componentes del equipo de desarrollo estaba limitado a un máximo
de ocho personas, quedando el equipo finalmente formado por siete personas.

El tiempo para desarrollar la aplicación vence el día 31 de enero de 2011.

\subsection{Entregables del proyecto}

\subsection{Resumen de la planificación y el presupuesto}

\section{Evolución del plan}
