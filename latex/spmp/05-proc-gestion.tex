\section{Plan de inicio del proyecto}

\subsection{Plan de estimación}

Para la estimación del proyecto, éste se ha estructurado en una serie de tareas
y se ha estimado un esfuerzo para cada tarea basado en la experiencia personal
del equipo de desarrollo en proyectos de características similares. Ese esfuerzo
estará expresado en horas y representará el tiempo necesario por una persona
para realizar dicha tarea.

Para la planificación del calendario del proyecto se ha tenido en cuenta un
esfuerzo de una hora diaria de media a la semana por cada integrante del grupo,
esto es, un total de 5 horas semanales.

\subsection{Plan de personal}

El grupo está integrado por un jefe de proyecto y seis desarrolladores. Éstos
forman parte del grupo desde su formación al inicio de la asignatura y
permanecerán en él durante todo el curso.

\subsection{Plan de adquisición de recursos}

En el desarrollo del proyecto se hará uso de dos aplicaciones fundamentales:
\href{http://www.eclipse.org}{Eclipse} y
\href{http://www.visual-paradigm.com}{Visual Paradigm}.
Ambas aplicaciones se pueden obtener libremente de sus respectivos sitios web,
y sólo en el caso de Visual Paradigm hará falta una licencia, que será
proporcionada por los profesores de la asignatura.

Se creará una lista de correo en \href{http://groups.google.com}{Google Groups}
para las comunicaciones internas del grupo. Todos los integrantes del grupo,
así como los profesores de la asignatura, recibirán una invitación a la lista
tan pronto como sea creada.

El trabajo de desarrollo se llevará a cabo en dos repositorios Git, uno dedicado
a la documentación del proyecto y otro al proyecto en sí. Ambos estarán
hospedados en \href{http://github.com}{GitHub}, donde deberán registrarse todos
los integrantes del grupo.

\subsection{Plan de formación de personal}

Los integrantes del grupo recibirán por parte de los profesores de la
asignatura a lo largo del curso los conocimientos y la ayuda necesaria para
llevar a cabo este proyecto.
