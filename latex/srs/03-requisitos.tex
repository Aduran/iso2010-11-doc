\section{Requisitos de interfaces externas}

\section{Requisitos funcionales}

\subsection{Registrarse}

{\footnotesize
\begin{tabularx}{0.95\textwidth}{p{0.2\textwidth}|X}

\textbf{Caso de uso} & Registrarse \\

\textbf{Precondición} & Ninguna \\

\textbf{Escenario general} & \begin{enumerate}
\item El usuario selecciona registrarse en la ventana de inicio.
\item La ventana principal muestra la ventana de registro.
\item El usuario introduce sus datos en la ventana de registro.
\item La ventana de registro verifica que los datos sean válidos y no habilita
el botón de confirmación hasta que lo sean.
\item El usuario pulsa el botón de confirmación.
\item La ventana de registro pide al gestor de usuarios que cree un nuevo
jugador con los datos indicados.
\item El gestor de usuarios pide al servidor que cree un nuevo jugador mediante
el proxy.
\item El servidor confirma al gestor de usuarios la creación del jugador.
\item La ventana de inicio informa al usuario del éxito de la operación.
\end{enumerate} \\

\textbf{Poscondición} & Un nuevo usuario existe en el sistema.

\end{tabularx}
}

\subsection{Iniciar sesión}

{\footnotesize
\begin{tabularx}{0.95\textwidth}{p{0.2\textwidth}|X}

\textbf{Caso de uso} & Iniciar sesión \\

\textbf{Precondición} & Usuario registrado \\

\textbf{Escenario general} & \begin{enumerate}
\item El usuario introduce su nombre y contraseña en la ventana de inicio.
\item La ventana de inicio verfica la validez de los datos.
\item El usuario pulsa el botón ``Iniciar sesión''.
\item El gestor de usuarios pide al servidor que valide al usuario.
\item El servidor confirma al gestor de usuarios que el usuario es válido.
\item El gestor de usuarios crea una sesión.
\item Se cierra la ventana de inicio y se abre la ventana principal.
\end{enumerate} \\

\textbf{Poscondición} & Se ejecuta el caso de uso ``Ver lista de partidas''
\\ \\

\textbf{Escenario alternativo} & Puede ocurrir que los datos de acceso no sean
correctos, en cuyo caso se aborta la operación y se vuelve al estado inicial.

\end{tabularx}
}

\subsection{Cerrar sesión}

{\footnotesize
\begin{tabularx}{0.95\textwidth}{p{0.2\textwidth}|X}

\textbf{Caso de uso} & Cerrar sesión \\

\textbf{Precondición} & Sesión iniciada \\

\textbf{Escenario general} & \begin{enumerate}
\item El usuario pulsa el botón ``Cerrar sesión'' en la ventana principal.
\item La ventana principal pide al gestor de usuarios que cierre la sesión.
\item El gestor de usuarios pide al servidor que cierre la sesión.
\item El servidor comunica al gestor de usuarios que ha cerrado la sesión.
\item Se cierra la ventana principal y se abre la ventana de inicio.
\end{enumerate} \\

\textbf{Poscondición} & El usuario deja de estar activo en el servidor.

\end{tabularx}
}

\subsection{Crear una nueva partida}
{\footnotesize
\begin{tabularx}{0.95\textwidth}{p{0.2\textwidth}|X}

\textbf{Caso de uso} & Crear una nueva partida \\

\textbf{Precondición} & El usuario debe estar logeado \\

\textbf{Escenario general} & \begin{enumerate}
\item El usuario selecciona de la ventana principal crear una nueva partida.
\item La ventana principal muestra la ventana de crear partida.
\item El usuario selecciona las características de la partida.
\item La ventana crear partida habilita el botón \emph{crear} cuando los datos
son correctos.
\item El usuario pulsa el botón crear.
\item La ventana crear partida notifica al gestor de partidas que un usuario
desea crear una partida.
\item El gestor de partidas pide al servidor mediante el proxy que cree una
partida.
\item El proxy le indica al gestor de partidas que se ha creado una nueva
partida.
\item El gestor modifica la ventana principal con el nuevo estado de las
partidas.
\item La ventana principal oculta la ventana de crear partidas.
\item La ventana principal comunica al jugador que se ha creado una nueva
partida.
\end{enumerate} \\

\textbf{Poscondición} & En el sistema hay una nueva partida y se ejecuta el caso
 de uso ``Unirse a partida''.\\ \\

\textbf{Escenario alternativo} & \begin{enumerate}
\item El usuario selecciona de la ventana principal crear una nueva partida.
\item La ventana principal muestra la ventana de crear partida.
\item El usuario selecciona las características de la partida.
\item La ventana crear partida habilita el botón \emph{crear} cuando los datos
son correctos.
\item El usuario pulsa el botón crear.
\item La ventana crear partida notifica al gestor de partidas que un usuario
desea crear una partida.
\item El gestor de partidas pide al servidor mediante el proxy que cree una
partida.
\item El proxy le indica al gestor de partidas que no se ha podido crear la
partida.
\item La ventana principal oculta la ventana de crear partidas.
\item La ventana principal comunica al jugador que no se ha podido crear la
partida.
\end{enumerate} \\

\end{tabularx}
}


\subsection{Unirse a una partida}
{\footnotesize
\begin{tabularx}{0.95\textwidth}{p{0.2\textwidth}|X}

\textbf{Caso de uso} & Unirse a partida \\

\textbf{Precondición} & El usuario debe estar logeado \\

\textbf{Escenario general} & \begin{enumerate}
\item El usuario selecciona de la lista de partidas disponibles de la ventana
principal.
\item La ventana notifica al gestor de usuario que el usuario desea unirse a la
partida seleccionada.
\item El gestor de usuarios se comunica con el servidor para indicarle que un
usuario quiere unirse a una partida mediante el proxy.
\item El proxy notifica al gestor que el usuario se ha unido a la partida.
\item El gestor de usuarios actualiza la lista de partidas del jugador.
\item El gestor modifica la ventana principal con el nuevo estado de las
partidas.
\end{enumerate} \\

\textbf{Poscondición} & En la partida seleccionada hay un nuevo jugador y se
ejecuta el caso de uso ``Conectarse a una partida''.\\

\textbf{Escenario alternativo} & \begin{enumerate}
\item El usuario selecciona de la lista de partidas disponibles de la ventana
principal.
\item La ventana notifica al gestor de usuario que el usuario desea unirse a la
partida seleccionada.
\item El gestor de usuarios se comunica con el servidor para indicarle que un
usuario quiere unirse a una partida mediante el proxy.
\item El proxy notifica al gestor que el usuario no se ha podido unir a la
partida ya que no hay sitio disponible.
\item El gestor de usuarios actualiza la lista de partidas disponibles.
\item El gestor modifica la ventana con el nuevo estado de las partidas.
\end{enumerate}

\end{tabularx}
}

\subsection{Conectarse a una partida}
{\footnotesize
\begin{tabularx}{0.95\textwidth}{p{0.2\textwidth}|X}

\textbf{Caso de uso} & Conectarse a una partida \\

\textbf{Precondición} & El jugador debe haber iniciado sesión y haberse unido a
la partida.\\

\textbf{Escenario general} & \begin{enumerate}
\item El usuario selecciona una partida dentro de la lista de partidas a las que
se ha unido ese usuario.
\item La ventana principal pide al gestor de partidas que inicie la partida
seleccionada.
\item El gestor de partidas pide al servidor la información de la partida
solicitada mediante el proxy.
\item El servidor envia la información correspondiente de la partida indicada al
gestor de partidas.
\item El gestor de partidas abre la nueva ventana de juego con la información
recibida de la partida.

\end{enumerate} \\

\textbf{Poscondición}

\end{tabularx}
}

\subsection{Desconectarse de una partida}
{\footnotesize
\begin{tabularx}{0.95\textwidth}{p{0.2\textwidth}|X}

\textbf{Caso de uso} & Desconectarse de una partida \\

\textbf{Precondición} & Estar conectado a una partida.\\

\textbf{Escenario general} & \begin{enumerate}
\item El usuario pulsa el botón cerrar partida en la ventana de juego.
\item La ventana de juego le indica al gestor de partidas que cierre la partida
en juego.
\item El gestor de partidas comunica al servidor el estado de la partida.
\item Se cierra la ventana de juego y se muestra la ventana principal.

\end{enumerate} \\

\textbf{Poscondición} & Se sale de la partida y aparece la ventana principal.

\end{tabularx}
}

\subsection{Ver la lista de partidas}

{\footnotesize
\begin{tabularx}{0.95\textwidth}{p{0.2\textwidth}|X}

\textbf{Caso de uso} & Ver la lista de partidas \\

\textbf{Precondición} & Iniciar sesión \\

\textbf{Escenario general} & \begin{enumerate}
\item La ventana principal pide al gestor de partidas una lista con las
partidas libres disponibles y una lista con las partidas que el usuario esta
jugando.
\item El gestor de partidas pide al servidor ambas listas.
\item El servidor envía ambas listas de partidas al gestor de partidas.
\item La ventana principal muestra la lista de partidas disponibles y la lista
de partidas comenzadas.

\end{enumerate} \\

\textbf{Poscondición} & La lista de partidas esta disponible para el usuario.

\end{tabularx}
}

\subsection{Recibir notificación de un nuevo jugador}

{\footnotesize
\begin{tabularx}{0.95\textwidth}{p{0.2\textwidth}|X}

\textbf{Caso de uso} & Recibir notificación de un nuevo jugador \\

\textbf{Precondición} & Para recibir notificaciones cuando un jugador se une a
una partida, es necesario que el jugador que quiere recibirlas esté unido a la
partida y logueado. \\

\textbf{Escenario general} & El jugador está conectado en el sistema jugando a
esta partida.\begin{enumerate}

\item El servidor envia al gestor de partidas el nombre del nuevo jugador y la
partida a la que se quiere conectar.
\item El gestor de partidas pasa esta información a la ventana de juego.
\item La ventana de juego muestra un mensaje indicando: ``El jugador X se
ha conectado a la partida Y''.

\end{enumerate} \\

\textbf{Poscondición} & Se realiza el caso de uso: ''Recibir actualización de la
partida''. \\ \\

\textbf{Escenario alternativo} & El jugador está conectado en el sistema jugando
a otra partida. En este caso no se realia la poscondición.

% En el caso de que el usuario no esté conectado, la aplicación cliente no tiene
% que hacer nada, es el servidor el que tiene que enviar un correo electrónico
% al usuario.

% LAURA aqui se ha liado el merge...
%\textbf{Escenario general} & \begin{enumerate}
%\item
%TODO {``Bifurcación" del caso de uso}

%\end{enumerate} \\

%\textbf{Poscondición} & La lista de partidas esta disponible para el usuario.
%%%

\end{tabularx}
}

\subsection{Realizar un movimiento}

{\footnotesize
\begin{tabularx}{0.95\textwidth}{p{0.2\textwidth}|X}

\textbf{Caso de uso} & Realizar un movimiento \\

\textbf{Precondición} & Iniciar sesión, Unirse a una partida y Conectarse a una
partida \\

\textbf{Escenario general} & \begin{enumerate}
\item El usuario selecciona el origen, el destino del movimiento (el destino
lo controla otro jugador) y el número de tropas con las que va a atacar en la
ventana de juego.
\item La ventana de juego comprueba que el movimiento es legal.
\item La ventana de juego envía la información al gestor de partidas.
\item El gestor de partidas envía la información al servidor através del proxy.
\item El servidor devuelve la respuesta a la acción de ataque al gestor de
partidas.
\item El gestor de partidas pasa a la ventana de juego la respuesta.
\item La ventana de juego pregunta si se acepta la oferta (si no se acepta
el ataque).
\item El usuario elige la acción a realizar.
\item La ventana de juego envia la acción al gestor de partidas.
\item El gestor de partidas realiza los cálculos de la actualización de la
partida.
\item El gestor de partidas actualiza la ventana de juego.
\end{enumerate} \\

\textbf{Poscondición} & Se realiza el caso de uso: ''Enviar actualización de la
partida''.\\

\textbf{Escenario alternativo} & Si los territorios seleccionados no están
juntos la aplicación no permitirá realizar la acción.

\end{tabularx}
}

\subsection{Responder a acción enemiga}

{\footnotesize
\begin{tabularx}{0.95\textwidth}{p{0.2\textwidth}|X}

\textbf{Caso de uso} & Responder a acción enemiga \\

\textbf{Precondición} & Iniciar sesión, Unirse a una partida y Conectarse a una
partida \\

\textbf{Escenario general} & \begin{enumerate}
\item El servidor envía al gestor de partidas la información del ataque del
oponente.
\item El gestor de partidas pasa la información a la ventana de juego.
\item La ventana de juego pregunta al jugador que acción realizar (aceptar
ataque, dar soldados o dar gallifantes).
\item El usuario elige la acción en la ventana de juego.
\item La ventana de juego envía la información al gestor de partidas.
\item El gestor de partidas envía la respuesta al servidor por medio del proxy.

\end{enumerate} \\

\textbf{Poscondición} & Se realiza el caso de uso: ''Recibir actualización de la
partida''.

\end{tabularx}
}

\subsection{Comprar refuerzos}

{\footnotesize
\begin{tabularx}{0.95\textwidth}{p{0.2\textwidth}|X}

\textbf{Caso de uso} & Comprar refuerzos \\

\textbf{Precondición} & El usuario debe haberse unido a una partida y estar
conectado a ella. \\

\textbf{Escenario general} & \begin{enumerate}
\item El usuario pulsa sobre el botón comprar.
\item Se muestra al usuario una ventana con diferentes opciones.
\item El usuario introduce los datos necesarios y confirma la operación.
\item El gestor del juego recibe los datos y env\'ia la información al proxy.
\item El proxy notifica al gestor de juego que la compra se ha realizado con
\'exito.
\item El gestor de juego actualiza los datos del juego.
\end{enumerate} \\
\textbf{Postcondición} & Se ejecuta el caso de uso  ``Enviar actualización de
la partida''. \\
\textbf{Escenario alternativo} & \begin{enumerate}
\item El usuario pulsa sobre el botón comprar.
\item Se muestra al usuario una ventana con diferentes opciones.
\item El usuario introduce los datos necesarios y confirma la operación.
\item El gestor del juego recibe los datos, comprueba que se cumple la condición
de que el jugador tenga dinero suficiente, el jugador no tiene dinero
suficiente.
\item El gestor de juego pide a la ventana que muestre un mensaje con
información sobre el error producido al realizar la acción de compra.
\end{enumerate}\\
\end{tabularx}
}

\subsection{Enviar petición de alianza}

{\footnotesize
\begin{tabularx}{0.95\textwidth}{p{0.2\textwidth}|X}

\textbf{Caso de uso} & Enviar petición de alianza \\

\textbf{Precondición} & El usuario debe haberse unido a una partida \\

\textbf{Escenario general} & \begin{enumerate}
\item El usuario pulsa sobre el botón alianza.
\item Se muestra al usuario una ventana con diferentes jugadores.
\item El usuario selecciona uno de ellos y confirma la operación.
\item El gestor del juego recibe los datos y envía la información al proxy.
\item El proxy notifica al gestor de juego que la alianza se ha realizado con
éxito.
\item El gestor de juego actualiza los datos del juego.
\end{enumerate} \\
\textbf{Postcondición} & Mostrar correctamente los datos de juego una vez
actualizado y se ejecuta el caso de uso  ``Enviar actualización de
la partida''. \\ \\
\textbf{Escenario alternativo} & \begin{enumerate}
\item El usuario pulsa sobre el botón alianza.
\item Se muestra al usuario una ventana con diferentes jugadores.
\item El usuario selecciona uno de ellos y confirma la operación.
\item El gestor del juego recibe los datos, comprueba que se cumple la condición
que permite la alianza, pero el jugador seleccionado no es válido para una
alianza.
\item El gestor de juego pide a la ventana que muestre un mensaje con
información sobre el error producido al realizar la alianza.
\end{enumerate}\\
\end{tabularx}
}

\subsection{Responder a petición de alianza}

{\footnotesize
\begin{tabularx}{0.95\textwidth}{p{0.2\textwidth}|X}
\textbf{Caso de uso} & Responder a petición de alianza \\

\textbf{Precondición} & El usuario debe haberse unido a una partida \\

\textbf{Escenario general} & \begin{enumerate}
\item El proxy comunica al gestor de juego que existe una petición de alianza.
\item El gestor de juego pide a la ventana que muestre al jugador el mensaje
pertinente.
\item El usuario selecciona la acción oportuna.
\item El gestor del juego recibe los datos y envía la información al proxy.
\item El proxy notifica al gestor de juego que la alianza se ha realizado con
éxito.
\item El gestor de juego actualiza los datos del juego.
\end{enumerate} \\
\textbf{Postcondición} & Mostrar correctamente los datos de juego una vez
actualizado Se ejecuta el caso de uso  ``Enviar actualización de
la partida''. .
\end{tabularx}
}

\subsection{Romper alianza}

{\footnotesize
\begin{tabularx}{0.95\textwidth}{p{0.2\textwidth}|X}

\textbf{Caso de uso} & Romper alianza \\

\textbf{Precondición} & El usuario debe haberse unido a una partida y tener, al
menos, una alianza \\

\textbf{Escenario general} & \begin{enumerate}
\item El usuario pulsa sobre el botón alianza.
\item Se muestra al usuario una ventana con diferentes alianzas.
\item El usuario selecciona una de ellas y confirma la operación.
\item El gestor del juego recibe los datos y envía la información al proxy.
\item El proxy notifica al gestor de juego que la alianza se ha roto con éxito.
\item El gestor de juego actualiza los datos del juego.
\end{enumerate} \\

\textbf{Postcondición} & Mostrar correctamente los datos de juego una vez
actualizado. Se ejecuta el caso de uso  ``Enviar actualización de
la partida''. .

\end{tabularx}
}

\subsection{Enviar actualización de la partida}

{\footnotesize
\begin{tabularx}{0.95\textwidth}{p{0.2\textwidth}|X}

\textbf{Caso de uso} & Enviar actualización de la partida \\

\textbf{Precondición} & El usuario debe haberse unido a una partida \\

\textbf{Escenario general} & \begin{enumerate}
\item El usuario realiza una operación que modifica la partida.
\item El gestor de juego recopila todos los datos modificados y envía la
información al proxy.
\item El proxy notifica al gestor de juego que la actualización se ha realizado
con éxito.
\end{enumerate} \\
\textbf{Postcondición} & Los datos del cliente y del servidor están
sincronizados.

\end{tabularx}
}

\subsection{Recibir actualización de la partida}

{\footnotesize
\begin{tabularx}{0.95\textwidth}{p{0.2\textwidth}|X}

\textbf{Caso de uso} & Recibir actualización de la partida \\

\textbf{Precondición} & El usuario debe haberse unido a una partida \\

\textbf{Escenario general} & \begin{enumerate}
\item El proxy comunica al gestor de juego que la partida ha cambiado.
\item El gestor de juego recibe todos los datos modificados y pide a la ventana
que se actualice.
\end{enumerate} \\
\textbf{Postcondición} & Los datos del cliente y del servidor están
sincronizados.

\end{tabularx}
}

\subsection{Actualizar lista de partidas}


\section{Requisitos no funcionales}
\subsection{Estructura lógica de los datos}

\subsection{Estructura lógica de los datos}

\subsubsection{Entidad Datos de registro}
\begin{tabularx}{0.9\textwidth}{llX}
\hline
\textbf{Elemento} & \textbf{Tipo} & \textbf{Descripción} \\
\hline
Nombre & Texto & Nombre del usuario \\
eMail & Texto & Dirección de correo electrónico \\
Contraseña & Texto & Contraseña en el sistema \\
Conf. Contraseña & Texto & Confirmación de contraseña \\
\hline
\end{tabularx}



\section{Otros requisitos}
\subsection{Requisitos de la interfaz de usuario}
\subsubsection{Ventana de inicio}
{\footnotesize

 \textbf{Descripción:} Da la bienvenida al usuario.\\ Debe permitir al usuario
ejecutar el
caso de uso ``Iniciar sesión'' e introducir los siguientes datos: \\

\begin{tabularx}{0.9\textwidth}{llX}
\hline
\textbf{Dato} & \textbf{Tipo} & \textbf{Descripción} \\
\hline
NombreUsuario & Texto & Nombre del usuario \\
Contraseña & Texto & Contraseña en el sistema \\
\hline
\end{tabularx}

\subsubsection{Ventana de registro}
{\footnotesize



 \textbf{Descripción:} Permite registrarse al usuario.\\
Es necesario que
permita al usuario introducir los siguientes datos: \\

\begin{tabularx}{0.9\textwidth}{llX}
\hline
\textbf{Dato} & \textbf{Tipo} & \textbf{Descripción} \\
\hline
Nombre & Texto & Nombre del usuario \\
eMail & Texto & Dirección de correo electrónico \\
Contraseña & Texto & Contraseña en el sistema \\
Conf. Contraseña & Texto & Confirmación de contraseña \\
\hline
\end{tabularx}

}

\subsubsection{Ventana principal}
{\footnotesize

 \textbf{Descripción:}  Mostrará al usuario las partidas disponibles y las
partidas en
juego.\\
Es necesario que
permita al usuario ejecutar los siguientes casos de uso: \\
\begin{enumerate}
\item Crear una partida.
\item Unirse a una partida.
\item Conectarse a una partida.
\item Ver lista de partidas.

\end{enumerate} 
}

\subsubsection{Ventana Crear partida}
{\footnotesize

 \textbf{Descripción:}  Permite al usuario crear una nueva partida.\\
Es necesario que
permita al usuario introducir los siguientes datos: \\

\begin{tabularx}{0.9\textwidth}{llX}
\hline
\textbf{Dato} & \textbf{Tipo} & \textbf{Descripción} \\
\hline
Nombre & Texto & Nombre de la partida \\
Días de juego & Fecha & Días en los que se jugará la partida\\
Hora de Inicio de Juego & Hora & Hora en la que empieza o continua la partida
los días indicados\\
Hora de Fin de Juego & Hora & Hora en la que termina la partida los días
indicados\\
\hline
\end{tabularx}
}

\subsubsection{Ventana de juego}
{\footnotesize



 \textbf{Descripción:}  Muestra al usuario todos los detalles de una partida y
permite
jugar en ella. Deberá mostrar el tablero de juego con la asignación de
territorios.\\

Es necesario que
permita al usuario ejecutar los siguientes casos de uso: \\
\begin{enumerate}
\item Desconectarse de una partida. 
\item Realizar un movimiento. 
\item Responder acción enemiga.
\item Comprar refuerzos.
\item Enviar petición de alianza.
\item Responder a petición de alianza.
\item Romper alianza.

\end{enumerate} 


}