\section{Requisitos de interfaces externas}

\section{Requisitos funcionales}

\subsection{Registrarse}

{\footnotesize
\begin{tabularx}{0.95\textwidth}{r|X}

\textbf{Caso de uso} & Registrarse \\

\textbf{Precondición} & Ninguna \\

\textbf{Escenario general} & \begin{enumerate}
\item El usuario selecciona registrarse en la ventana de inicio.
\item La ventana principal muestra la ventana de registro.
\item El usuario introduce sus datos en la ventana de registro.
\item La ventana de registro verifica que los datos sean válidos y no habilita
el botón de confirmación hasta que lo sean.
\item El usuario pulsa el botón de confirmación.
\item La ventana de registro pide al gestor de usuarios que cree un nuevo
jugador con los datos indicados.
\item El gestor de usuarios pide al servidor que cree un nuevo jugador mediante
el proxy.
\item El servidor confirma al gestor de usuarios la creación del jugador.
\item La ventana de inicio informa al usuario del éxito de la operación.
\end{enumerate} \\

\textbf{Poscondición} & Un nuevo usuario existe en el sistema.

\end{tabularx}
}

\subsection{Iniciar sesión}

\subsection{Cerrar sesión}

\subsection{Crear una nueva partida}
{\footnotesize
\begin{tabularx}{0.95\textwidth}{r|X}

\textbf{Caso de uso} & Crear una nueva partida \\

\textbf{Precondición} & El usuario debe estar logeado \\

\textbf{Escenario general} & \begin{enumerate}
\item El usuario selecciona de la ventana de juego crear una nueva partida.
\item La ventana de juego muestra la ventana de crear partida.
\item El usuario selecciona las características de la partida.
\item La ventana crear partida habilita el botón \emph{crear} cuando los datos son correctos.
\item El usuario pulsa el botón crear.
\item La ventana crear partida notifica al gestor de partidas que un usuario desea crear una partida. 
\item El gestor de partidas pide al servidor mediante el proxy que cree una partida.
\item El proxy le indica al gestor de partidas que se ha creado una nueva partida.
\item El gestor modifica la ventana de juego con el nuevo estado de las partidas.  
\item La ventana de juego oculta la ventana de crear partidas.
\item La ventana principal comunica al jugador que se ha creado una nueva partida.
\end{enumerate} \\

\textbf{Poscondición} & En el sistema hay una nueva partida.

\end{tabularx}
}


\subsection{Unirse a una partida}
{\footnotesize
\begin{tabularx}{0.95\textwidth}{r|X}

\textbf{Caso de uso} & Unirse a partida \\

\textbf{Precondición} & El usuario debe estar logeado \\

\textbf{Escenario general} & \begin{enumerate}
\item El usuario selecciona de la lista de partidas disponibles de la ventana de juego.
\item La ventana notifica al gestor de usuario que el usuario desea unirse a la partida seleccionada.
\item El gestor de usuarios se comunica con el servidor para indicarle que un usuario quiere unirse a una partida mediante el proxy.
\item El proxy notifica al gestor que el usuario se ha unido a la partida.
\item El gestor de usuarios actualiza la lista de partidas del jugador.
\item El gestor modifica la ventana de juego con el nuevo estado de las partidas. 
\end{enumerate} \\

\textbf{Poscondición} & En la partida seleccionada hay un nuevo jugador.\\ \\

\textbf{Escenario alternativo} & \begin{enumerate}
\item El usuario selecciona de la lista de partidas disponibles de la ventana de juego.
\item La ventana notifica al gestor de usuario que el usuario desea unirse a la partida seleccionada.
\item El gestor de usuarios se comunica con el servidor para indicarle que un usuario quiere unirse a una partida mediante el proxy.
\item El proxy notifica al gestor que el usuario no se ha podido unir a la partida ya que no hay sitio disponible.
\item El gestor de usuarios actualiza la lista de partidas disponibles.
\item El gestor modifica la ventana con el nuevo estado de las partidas. 
\end{enumerate}

\end{tabularx}
}

\subsection{Conectarse a una partida}

\subsection{Desconectarse de una partida}

\subsection{Ver la lista de partidas}

\subsection{Recibir información de una partida}

\subsection{Recibir notificación de un nuevo jugador}

\subsection{Realizar un movimiento}

\subsection{Comprar refuerzos}

\subsection{Enviar petición de alianza}

\subsection{Recibir petición de alianza}

\subsection{Romper alianza}

\subsection{Enviar actualización de la partida}

\subsection{Recibir actualización de la partida}

\section{Requisitos no funcionales}

\subsection{Estructura lógica de los datos}

\subsubsection{Entidad Datos de registro}
\begin{tabularx}{0.9\textwidth}{llX}
\hline
\textbf{Elemento} & \textbf{Tipo} & \textbf{Descripción} \\
\hline
Nombre & Texto & Nombre del usuario \\
eMail & Texto & Dirección de correo electrónico \\
Contraseña & Texto & Contraseña en el sistema \\
Conf. Contraseña & Texto & Confirmación de contraseña \\
\hline
\end{tabularx}
