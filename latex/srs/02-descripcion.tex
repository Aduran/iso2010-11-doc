\section{Perspectiva del producto}

\textit{La conquista del mundo} pretende ser un juego multijugador similar al
juego de mesa \textit{Risk} en el que varios jugadores compiten por dominar el
mundo.

Este proyecto se centra en realizar un cliente para este juego.

\subsection{Interfaces con otros sistemas}

La aplicación estará conectada directamente con el servidor del juego, quien a
su vez servirá de pasarela hacia los otros clientes. Esta comunicación será
bidireccional, de manera que el cliente notificará al servidor sus acciones, y
el servidor notificará al cliente las acciones de los demás clientes.

\subsection{Interfaces de usuario}

La aplicación proporcionará una interfaz gráfica de usuario el jugador. Antes
de poder realizar ninguna acción, el jugador deberá estar validado con el
servidor. Acto seguido se mostrará una lista de partidas en las que el jugador
pueda participar. Las partidas se jugarán sobre un mapa del mundo interactivo.

\subsection{Interfaces hardware}

La aplicación no se comunica con ningún dispositivo hardware.

\subsection{Interfaces software}

La aplicación estará escrita en el lenguaje de programación Java y por tanto
necesitará ejecutarse sobre una máquina virtual de Java. Sin embargo, la
aplicación no necesita ningún otro servicio del sistema operativo.

\subsection{Interfaces de comunicaciones}

La aplicación se comunica con el servidor a través del \textit{middleware} RMI,
que usa TCP/IP como capa de transporte.

\subsection{Restricciones de memoria}

La aplicación está pensada para ejecutarse sobre ordenadores personales de uso
común. La cantidad de memoria que tienen los ordenadores personales hoy en día
debería ser ampliamente suficiente para ejecutar sin problemas de rendimiento
la aplicación.

\subsection{Modos de operación}

En el momento en el que el usuario se valida con el servidor, la aplicación
tiene dos modos de ejecución: 1) uno en el que el jugador puede ver una lista
con las partidas disponibles, y 2) otro en el que el jugador se encuentra
conectado a una partida específica.

\section{Funciones del producto}

En primer lugar, la aplicación presentará al usuario una ventana desde la cual
podrá acceder al sistema. Para ello deberá tener una cuenta en el sistema, por
lo que deberá permitir al usuario crear una nueva cuenta si no tuviera.

Una vez haya accedido el usuario, la aplicación deberá mostrarle una lista de
partidas. Estas partidas podrán ser, bien partidas en las que haya al menos un
territorio libre y puede unirse a ellas, o partidas a las que ya se haya unido
previamente. Desde esta lista, el usuario podrá conectarse a una de sus partidas
para jugar.

Cuando el usuario se una a una partida, el cliente descargará los datos desde
el servidor y le mostrará el tablero de juego. El tablero consiste en un mapa
del mundo dividido en varias regiones por cada continente.

Desde el tablero de juego, el usuario podrá realizar varias acciones:
movimientos militares, misiones de espionaje, adquisición de refuerzos, y
acciones diplomáticas.

Los movimientos militares sólo se podrán realizar entre dos regiones
adyacentes. Se considerarán como adyacentes aquellas regiones que compartan
parte de su frontera, o estando separadas por mar, aparezcan unidas por una
línea discontinua. Un usuario podrá mover tropas entre regiones propias,
invadir regiones que aún estén vacías, o atacar regiones de otros usuarios.

Mediante las misiones de espioanaje, el usuario podrá ser capaz de desvelar
datos sobre otros jugadores que desconoce, como la cantidad de tropas en una
determinada región, o los recursos de que dispone.

Durante una partida, el usuario podrá adquirir nuevas tropas y agregarlas a una
región determinada. Habrá varios tipos de tropas: soldados, cañones, misiles y
antimisiles, y espías. El usuario también podrá pactar con otros usuarios y
formar así alianzas.

\section{Características de los usuarios}

Puesto que la aplicación se trata de un sencillo juego de estrategia, cualquier
persona estaría cualificada para usar correctamente la aplicación, suponiendo
que tenga los conocimientos elementales sobre el uso de un ordenador a nivel de
usuario.

\section{Restricciones}

La aplicación debe trabajar en sintonía con el resto de clientes y el servidor
del juego. Para ello deberá definirse una interfaz común para todos, lo que
condicionará el diseño de algunos componentes de la aplicación.

La tecnología de comunicaciones debe ser RMI, y ésta sólo se encuentra como
parte de la máquina virtual de Java, lo que hace que el lenguaje de programación
deba ser \textit{de facto} Java.
