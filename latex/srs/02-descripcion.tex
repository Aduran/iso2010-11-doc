\section{Perspectiva del producto}

\textit{La conquista del mundo} pretende ser un juego multijugador similar al
juego de mesa \textit{Risk} en el que varios jugadores compiten por dominar el
mundo.

Este proyecto se centra en realizar un cliente para este juego.

\subsection{Interfaces con otros sistemas}

La aplicación estará conectada directamente con el servidor del juego, quien a
su vez servirá de pasarela hacia los otros clientes. Esta comunicación será
bidireccional, de manera que el cliente notificará al servidor sus acciones, y
el servidor notificará al cliente las acciones de los demás clientes.

\subsection{Interfaces de usuario}

La aplicación proporcionará una interfaz gráfica de usuario el jugador. Antes
de poder realizar ninguna acción, el jugador deberá estar validado con el
servidor. Acto seguido se mostrará una lista de partidas en las que el jugador
pueda participar. Las partidas se jugarán sobre un mapa del mundo interactivo
sobre.

\subsection{Interfaces hardware}

La aplicación no se comunica con ningún dispositivo hardware.

\subsection{Interfaces software}

La aplicación estará escrita en el lenguaje de programación Java y por tanto
necesitará ejecutarse sobre una máquina virtual de Java. Sin embargo, la
aplicación no necesita ningún otro servicio del sistema operativo.

\subsection{Interfaces de comunicaciones}

La aplicación se comunica con el servidor a través del \textit{middleware} RMI,
que usa TCP/IP como capa de transporte.

\subsection{Restricciones de memoria}

La aplicación está pensada para ejecutarse sobre ordenadores personales de uso
común. La cantidad de memoria que tienen los ordenadores personales hoy en día
debería ser ampliamente suficiente para ejecutar sin problemas de rendimiento
la aplicación.

\subsection{Modos de operación}

En el momento en el que el usuario se valida con el servidor, la aplicación
tiene dos modos de ejecución: 1) uno en el que el jugador puede ver una lista
con las partidas disponibles, y 2) otro en el que el jugador se encuentra
conectado a una partida específica.
